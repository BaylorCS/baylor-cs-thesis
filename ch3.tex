\chapter{Floats}

\section{Figures}

Figure~\ref{figure_example1} has a short caption, and
Figure~\ref{figure_example2} has a longer caption, demonstrating the required
single-spacing.

\begin{figure}[ht]
\caption{This is a caption for this figure}
\label{figure_example1}
\centering
\includegraphics[width=1in]{baylor}
\end{figure}

\begin{figure}[ht]
\caption[The short table of contents version]{An example of a longer figure
caption that spans multiple lines and has a corresponding short version for the
table of contents.}
\label{figure_example2}
\centering
\includegraphics[width=0.25\textwidth]{baylor}
\end{figure}

\section{Tables}

Table~\ref{table_fruit} and Table~\ref{table_silly} demonstrate tables. Table
captions differ slightly from figure captions, in that they are \textit{always}
supposed to be centered, even if they use multiple lines.

\begin{table}[ht]
\caption[Fruits by color]{\centering Fruits listed by their color. Note that
captions differ from figure captions in that they are \textit{always} supposed
to be centered, even if they use multiple lines.}
\label{table_fruit}
\begin{center}
\begin{small}
\begin{tabular}{rl}
    \hline
    \abovespace\belowspace
    Fruit & Color  \\  \hline
    \abovespace Orange & Orange \\
    Blue & Blueberry \\
    Red & Cherry \\
    Green & Apple \\
    Yellow & Banana \\
    Purple & Eggplant \belowspace \\
    \hline
\end{tabular}
\end{small}
\end{center}
\end{table}

\begin{table}[ht]
\caption[A silly table]{Another table example}
\label{table_silly}
\begin{center}
\begin{small}
\begin{tabular}{ccc}
    \hline
    \abovespace\belowspace
    A & B & C \\  \hline
    \abovespace 1 & 2 & 3 \\
    4 & 5 & 6 \\
    7 & 8 & 9 \belowspace \\
    \hline
\end{tabular}
\end{small}
\end{center}
\end{table}
